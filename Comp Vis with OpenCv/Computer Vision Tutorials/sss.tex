% LaTeX resume using res.cls

\documentclass[margin]{res}

\makeatletter
\def\@classoptionslist{<class options except `margin` OR empty>}
\makeatother
%\usepackage{helvetica} % uses helvetica postscript font (download helvetica.sty)
%\usepackage{newcent}   % uses new century schoolbook postscript font 
\setlength{\textwidth}{5.4in} % set width of text portion
\usepackage[left=0.9in,top=0.2in,right=2.4in,bottom=0.2in]{geometry}
\usepackage{hyperref}
\usepackage{csquotes}
\MakeOuterQuote{"}
\begin{document}


% Center the name over the entire width of resume:
 \moveleft.5\hoffset\centerline{\LARGE\bf Ravi Prakash}

% Draw a horizontal line the whole width of resume:
 \moveleft\hoffset\vbox{\hrule width\resumewidth height 1pt}\smallskip
% address begins here
% Again, the address lines must be centered over entire width of resume:

 \moveleft.5\hoffset\centerline{\large{Junior Undergraduate}}
 \moveleft.5\hoffset\centerline{\large{School of Electronics \& Communication Engineering}}
 \moveleft.5\hoffset\centerline{\large{Shri Mata Vaishno Devi University}}
\moveleft.5\hoffset\centerline{\large{Katra 182 320, Jammu \& Kashmir, India}}
\moveleft.5\hoffset\centerline{\large{\normalsize{\textbf{Mobile:} +91 - 9797296954
\textbf{Email:} \texttt{16bec047@smvdu.ac.in}}}}
 \moveleft\hoffset\vbox{\hrule width\resumewidth height 1pt}
\begin{resume}
\section{OBJECTIVE}  To contribute to the field of Artificial Intelligence, Computer Vision and work for cutting-edge research and innovation with skills acquired while pursuing Engineering in a stimulating environment of a reputed organization while acquiring real life experience and exposure.
 

\section{EDUCATION} {\sl \textbf{Bachelor of Technology}}, Electronics \& Communication Engineering \hfill 2016-20\\
                      % \sl will be bold italic in New Century Schoolbook (or
	              % any postscript font) and just slanted in
		      %	Computer Modern (default) font
                SMVD Univ., J \& K India \hfill \textbf{CGPA}: 7.45/10 \\
{\sl \textbf{Senior Secondary Certificate Examination (CBSE)}} \hfill 2015\\
                      % \sl will be bold italic in New Century Schoolbook (or
	              % any postscript font) and just slanted in
		      %	Computer Modern (default) font
                Oxford Seniour Secondary SChool, Bihar, India \hfill
                \textbf{Percentage}: 94.00\%\\
{\sl \textbf{Secondary School Examination (CBSE)}} \hfill 2013\\
                      % \sl will be bold italic in New Century Schoolbook (or
	              % any postscript font) and just slanted in
		      %	Computer Modern (default) font
                D.A.V. Public School, Bihar, India \hfill               \textbf{C.G.P.A.}: 10/10
          
 
 
\section{AREAS OF \\ INTEREST} {\sl \textbf{Deep Learning:}} Deep Neural Networks, Generative Adversarial Networks , Object Tracking Algorithms, Pose Estimation and  their implementation in  A.I. based Robotics.\\
\\
                {\sl \textbf{Embedded Systems and Robotics:}} Working with embedded systems like ARM Cortex,development boards like Raspberry Pi and Arduino, Implementation of Deep Learning Algorithms in the same, Autonomous Driving .\\
                \\
{\sl \textbf{Internet of Things:}} Cloud Databases,Smart Devices,Use of Pooling of Cars to create a network of Autonomous Cars learning simultaneously.




\section{RELEVANT \\ PROJECTS} 
\textbf{Autonomous Driving Car } \hfill (Ongoing)\\
The main aim is to design an Autonomous Car using Deep Learning, Computer Vision and Robotics(including Embedded Systems) (in PYTHON(tensorflow,keras),CPP).\\ 
\textsl{Project Supervisor: Dr. Rakesh Jha }\textbf{((Asst. Prof, SMVDU, J and K  )}\vspace{0.15in}\\ 
\textbf{AnnaData Android Ml based Application } \hfill (March 2018)\\
The main aim was to design an interactive  Android App to assist farmers.The app's main features included Weather Pattern Analysis and provide GUI based info to the farmers.\\ 
\textsl{Project Supervising Team: }\textbf{((Rajasthan Hackathon,DoITC,Government of Rajasthan )}\vspace{0.15in}\\ 
\textbf{Realtime Emotion Detection} \hfill January 2019\\
The main objective of the Project was to create a system which can differentiate between various emotions.(in PYTHON(tensorflow,opencv),CPP)\

\textbf{Portable Weather Station} \hfill (December 2018)\\
The main aim of the project was to develop an I.O.T. based weather station through which anyone can get realtime and very precise data related to the surrounding area's weather. Also, using efficient M.L., we analysed and tried to predict the nearby patterns.(in PYTHON,CPP,NodeMCU)
\textsl{Project Supervisor: Mr. A.K. Bhardwaj} \textbf{(Asst. Prof, SMVDU, J \& K)}

\textbf{Realtime monitoring of 3-D space and object  estimation} \hfill (October 2018)\\
Here, the aim was to design and implement a system which could scan a 3-d space and generate a UI based map of the same. Also,it could locate any object's co-ordinates.\\
\textsl{Project Supervisor: Mr. Shashi Bhushan Kotwal} \textbf{(Asst. Prof, SMVDU, J \& K)}


%\section{PUBLICATION}
%\textbf{Sudhakar Kumar}, Tushar Kanti Das and Rabul Hussain Laskar, "Significance of Acoustic Features for designing an Emotion Classification System, " in 8\textsuperscript{th} International Conference on Electrical and Computer Engineering (ICECE) held at BUET, Bangladesh in Dec 2014. \hfill[Present status - \textbf{In Press}] 

\section{TECHNICAL \\ SKILLS}
\textsl{\textbf{Technical Softwares and Frameworks}} \hspace{9pt}: Tensorflow,Keras,Pytorch,Caffe,Android Studio,OpenCV,CUDA,MATLAB,Xilinx ISE (VHDL), NI Multisim,TINA,Keil.\\
\textsl{\textbf{Hardware}}  \hspace{55pt} : Raspberry Pi, ARM CORTEX,Arduino , NodeMCU\\
\textsl{\textbf{Languages}} \hspace{55pt}: Python,Java,Cpp,C,Assembly,Javscript,GO,Kotlin.



%\section{AWARDS AND RECOGNITIONS}
%\begin{itemize} \setlength\itemsep{-0.09em}
%\item Member of International Association of Engineers).\hfill{2019}
%\item Student Merit Cum Means Scholarship at SMVD University. \hfill{2013-14} 
%\item Foundation for Excellence Scholarship (SID No. 13727).\hfill{2012-16}
%\item Sahara India Scholarship.\hfill{2010}
%\item Certificate of Merit from CBSE for being among top 0.1\% students in SSE %\hfill{2009}
%\item Japan Tour as a JENESYS Candidate (sponsored by Japan Government).\hfill{2007}
%\end{itemize}

\section{MOOCS \\}
1. Convolutional Neural Nets\hfill \textbf{Stanford University,COURSERA}\\
2. Improving Deep Neural Networks  \hfill \textbf{Stanford University,COURSERA}\\
3. Professional Data Analytics \hfill \textbf{UDEMY}\\
4. Internet of Things and Embedded Systems \hfill \textbf{University of California}\\
5. Randomized Algorithms \hfill \textbf{Stanford University}\\
6. Neural Networks and Deep Learning
 \hfill \textbf{Stanford University}\\
7. Machine Learning \hfill \textbf{Stanford University,COURSERA}\\



\section{EXTRA-CURRICULAR ACTIVITIES}

\begin{itemize}  \setlength\itemsep{-0.00em}
\item Student Coordinator of the Mega Event(MAZE SOLVER ) at SMVD University's Technical festival.\hfill{2018}
\item Top 1000 students in English Olympiad 2011. \hfill {2011}
\item Lead a team of Developers at RAJASTHAN HACKATHON, a straight 36 hours coding challenge. \hfill{2018}
\item First in BOLLYMANIA at RESURGENCE 2K16-17 at SMVD University.\hfill{2017}
\item Participated in a  METHOD ACTING Workshop held in our Campus. \hfill{2017}
\item IKEN SCIENTIFICA Scientist Challenge national finalist, New Delhi  \hfill{2013}

\end{itemize}

\section{PERSONAL DETAILS}

{\sl{\textbf{Date of Birth}\hspace{56pt}}}: May 31, 1998\\
{\sl\textbf{Nationality}\hspace{68pt}}: Indian\\
{\sl\textbf{Permanent Address}\hspace{25pt}}: House number 32,Ward 46, Kanchan Nagar East,\\ \hspace*{46mm}Rambag Road,Post Office:- Ramna\\
 \hspace*{46mm}Distt. Muzaffarpur,\\\hspace*{45mm}
Bihar ,842002, India. \\
{\sl{\textbf{LinkedIn}\hspace{80pt}}}: https://www.linkedin.com/in/ravi-prakash-4b3a33151/\\
{\sl{\textbf{Github}\hspace{89pt}}}: https://github.com/ravi0531rp\\
\end{resume}
\end{document}




